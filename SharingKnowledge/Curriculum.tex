\PassOptionsToPackage{unicode=true}{hyperref} % options for packages loaded elsewhere
\PassOptionsToPackage{hyphens}{url}
%
\documentclass[
]{article}
\usepackage{lmodern}
\usepackage{amssymb,amsmath}
\usepackage{ifxetex,ifluatex}
\ifnum 0\ifxetex 1\fi\ifluatex 1\fi=0 % if pdftex
  \usepackage[T1]{fontenc}
  \usepackage[utf8]{inputenc}
  \usepackage{textcomp} % provides euro and other symbols
\else % if luatex or xelatex
  \usepackage{unicode-math}
  \defaultfontfeatures{Scale=MatchLowercase}
  \defaultfontfeatures[\rmfamily]{Ligatures=TeX,Scale=1}
\fi
% use upquote if available, for straight quotes in verbatim environments
\IfFileExists{upquote.sty}{\usepackage{upquote}}{}
\IfFileExists{microtype.sty}{% use microtype if available
  \usepackage[]{microtype}
  \UseMicrotypeSet[protrusion]{basicmath} % disable protrusion for tt fonts
}{}
\makeatletter
\@ifundefined{KOMAClassName}{% if non-KOMA class
  \IfFileExists{parskip.sty}{%
    \usepackage{parskip}
  }{% else
    \setlength{\parindent}{0pt}
    \setlength{\parskip}{6pt plus 2pt minus 1pt}}
}{% if KOMA class
  \KOMAoptions{parskip=half}}
\makeatother
\usepackage{xcolor}
\IfFileExists{xurl.sty}{\usepackage{xurl}}{} % add URL line breaks if available
\IfFileExists{bookmark.sty}{\usepackage{bookmark}}{\usepackage{hyperref}}
\hypersetup{
  pdftitle={Chương trình học},
  pdfauthor={Thanh Kim},
  pdfborder={0 0 0},
  breaklinks=true}
\urlstyle{same}  % don't use monospace font for urls
\usepackage[margin=1in]{geometry}
\usepackage{graphicx,grffile}
\makeatletter
\def\maxwidth{\ifdim\Gin@nat@width>\linewidth\linewidth\else\Gin@nat@width\fi}
\def\maxheight{\ifdim\Gin@nat@height>\textheight\textheight\else\Gin@nat@height\fi}
\makeatother
% Scale images if necessary, so that they will not overflow the page
% margins by default, and it is still possible to overwrite the defaults
% using explicit options in \includegraphics[width, height, ...]{}
\setkeys{Gin}{width=\maxwidth,height=\maxheight,keepaspectratio}
\setlength{\emergencystretch}{3em}  % prevent overfull lines
\providecommand{\tightlist}{%
  \setlength{\itemsep}{0pt}\setlength{\parskip}{0pt}}
\setcounter{secnumdepth}{-2}
% Redefines (sub)paragraphs to behave more like sections
\ifx\paragraph\undefined\else
  \let\oldparagraph\paragraph
  \renewcommand{\paragraph}[1]{\oldparagraph{#1}\mbox{}}
\fi
\ifx\subparagraph\undefined\else
  \let\oldsubparagraph\subparagraph
  \renewcommand{\subparagraph}[1]{\oldsubparagraph{#1}\mbox{}}
\fi

% set default figure placement to htbp
\makeatletter
\def\fps@figure{htbp}
\makeatother


\title{Chương trình học}
\author{Thanh Kim}
\date{16/08/2020}

\begin{document}
\maketitle

\hypertarget{chux1b0ux1a1ng-truxecnh-phuxe2n-tuxedch-dux1eef-liux1ec7u-huxe8-2020}{%
\section{Chương trình Phân tích dữ liệu Hè
2020}\label{chux1b0ux1a1ng-truxecnh-phuxe2n-tuxedch-dux1eef-liux1ec7u-huxe8-2020}}

\textbf{Khóa học ``Ứng dụng ngôn ngữ R trong phân tích dữ liệu Y sinh''}

\textbf{Đối tượng học}: Băng đảng YTCC

\textbf{Mục tiêu học}: Chuyển đổi từ Stata qua R, và một số tò mò khác

\textbf{Chương trình học}:

Bài 1: Giới thiệu R và R studio

Bài 2: Những trò cơ bản với R

Bài 3: Thực hành những trò cơ bản với R

Bài 4: Cuộc đời là một vòng lặp (Cơ bản)

Bài 5: Thực hành vòng lặp (Cơ bản)

Bài 6: Vệ sinh và tỉa tót (Dữ liệu)

Bài 7: Thực hành vệ sinh và tỉa tót (Dữ liệu)

Bài 8: Tìm hiểu, khám phá, mò mẫm (dữ liệu)

Bài 9: Thực hành tìm hiểu, khám phá, và mò mẫm

Bài 10: Có thể sinh sản nhiều ??!! (hay Reproducible Research)

Bài 11: Thực hành sinh sản nhiều ??!! (hay Reproducible Research)

Bài 12: Suy diễn, tưởng tượng (thống kê)

Bài 13: Thực hành suy diễn, tưởng tượng (thống kê)

Bài 14: Đầu quy là hồi tiên (Đầu tiên là hồi quy)

Bài 15: Thực hành đầu quy là hồi tiên

Thời gian dự kiến là 3 tháng. Có thể nhanh hoặc chậm hơn. Tuy nhiên mình
cứ thoải mái và học theo tiến độ của nhóm.

\textbf{Giáo trình học}

PPT Tiếng Việt.

\emph{Giáo trình được xây dựng từ những sách dưới đây. Mọi người có thể
tìm đọc. Tất cả đều miễn phí.}

Bài 1 - Bài 6: \url{https://leanpub.com/rprogramming}

Bài 8 - Bài 9: \url{https://leanpub.com/exdata}

Bài 9 - Bài 10: \url{https://leanpub.com/reportwriting}

Bài 12 - Bài 13:
\url{https://leanpub.com/LittleInferenceBook/read\#leanpub-auto-t-test-in-r}

Bài 14 - Bài 15: \url{https://leanpub.com/regmods}

\end{document}
